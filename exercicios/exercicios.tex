\documentclass[12pt]{article}
\usepackage[brazil]{babel}
\usepackage[utf8]{inputenc}
\usepackage{amsmath}
\usepackage{geometry}
\geometry{a4paper, margin=2cm}
\title{Gincana: Caçada Trigonométrica}
\date{}
\begin{document}
	
	\maketitle
	
	\section*{Estação 1 — Mesa de Ping-Pong}
	Encontre a largura da mesa de ping-pong, sabendo que $m = n$ e que o valor de $m$ será fornecido. Utilize a relação:
	\[
	h^2 = m \cdot n
	\]
	A largura da mesa corresponde à altura do triângulo formado na marcação do solo.
	
	\section*{Estação 2 — Placa no Muro}
	Uma placa está apoiada em um muro, formando um triângulo retângulo. A base mede 3 m. Sabendo a altura do muro (informada no local), determine a altura da placa utilizando o Teorema de Pitágoras.
	
	\section*{Estação 3 — Retângulo na Parede}
	Na figura representada na parede, $ABCD$ é um retângulo e os segmentos de reta $DE$ e $BF$ são perpendiculares à diagonal $AC$. Sendo $AB = X$ m e $BC = Y$ m, calcule o comprimento do segmento $EF$.
	
	\section*{Estação 4 — Círculo desenhado no chão}
	Um setor circular foi desenhado no chão, formando um triângulo com base e dois raios. Determine a área aproximada da região circular delimitada sabendo que $\pi = 3{,}14$ e que o ângulo será informado no local.
	
	\section*{Estação 5 — Mastro da Rede de Vôlei}
	Do topo do mastro da rede de vôlei, uma corda foi esticada até o chão, formando um triângulo retângulo. Sabendo o comprimento da base (desenhada no chão) e o ângulo de inclinação da corda (informado no local), determine a altura do mastro.
	
	\section*{Estação 6 — Portão de Ferro}
	No portão foi traçada uma diagonal com giz. Sabendo o comprimento da base do portão (informado no local) e o comprimento da diagonal, determine a altura do portão utilizando o Teorema de Pitágoras.
	
	\section*{Estação 7 — Poste com Cordas}
	Do topo de um poste, duas cordas foram fixadas até o chão, formando um ângulo reto no topo. Sabendo os valores de $m$ e $n$, calcule os comprimentos das duas cordas utilizando:
	\[
	c^2 = a \cdot n \quad \text{e} \quad b^2 = a \cdot m
	\]
	com $a = m + n$.
	
	\section*{Estação 8 — Mastro de Bandeira}
	Para descobrir a altura do mastro, foi desenhado um triângulo no chão com base fixa e ângulo informado no local. Determine a altura do mastro utilizando relações trigonométricas (seno, cosseno ou tangente).
	
	\section*{Estação 9 — Retângulo de Ferro}
	Na parede há um retângulo metálico com uma diagonal representada por um barbante. Com o auxílio de um transferidor, meça o ângulo formado pela diagonal e determine a área do retângulo.
	
\end{document}
